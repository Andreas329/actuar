\documentclass[x11names]{article}
\usepackage[T1]{fontenc}
\usepackage[utf8]{inputenc}
  \usepackage{amsmath,bm}
  \usepackage[round]{natbib}
  \usepackage[english]{babel}
  \usepackage[scaled=0.9]{helvet}
  \usepackage[sc]{mathpazo}
  \usepackage{framed,booktabs}
  \usepackage[inline,shortlabels]{enumitem}
  \usepackage[noae]{Sweave}

  %\VignetteIndexEntry{Loss distributions modeling}
  %\VignettePackage{actuar}

  \title{Inventory of continuous and discrete distributions provided
    in \pkg{actuar}}
  \author{Christophe Dutang \\ ISFA, Université Claude Bernard Lyon 1 \\[3ex]
    Vincent Goulet \\ École d'actuariat, Université Laval \\[3ex]
    Mathieu Pigeon \\ École d'actuariat, Université Laval}
  \date{}

  %% Colors
  \usepackage{xcolor}
  \definecolor{link}{rgb}{0,0.4,0.6}             % internal links
  \definecolor{url}{rgb}{0.6,0,0}                % external links
  \definecolor{citation}{rgb}{0,0.5,0}           % citations
  \definecolor{codebg}{named}{LightYellow1}      % R code background

  %% Hyperlinks
  \usepackage{hyperref}
  \hypersetup{%
    pdfauthor={Vincent Goulet},
    colorlinks = {true},
    linktocpage = {true},
    urlcolor = {url},
    linkcolor = {link},
    citecolor = {citation},
    pdfpagemode = {UseOutlines},
    pdfstartview = {Fit},
    bookmarksopen = {true},
    bookmarksnumbered = {true},
    bookmarksdepth = {subsubsection}}

  %% Sweave environments redefined to have a color background (using
  %% environment snugshade* of package framed).
  \DefineVerbatimEnvironment{Sinput}{Verbatim}{}
  \DefineVerbatimEnvironment{Soutput}{Verbatim}{}
  \fvset{listparameters={\setlength{\topsep}{0pt}}}
  \renewenvironment{Schunk}{%
    \setlength{\topsep}{0pt}
    \colorlet{shadecolor}{codebg}
    \begin{snugshade*}}%
    {\end{snugshade*}}

  %% Compact, sans label itemize environment for the appendices.
  \setlist[itemize]{label={},leftmargin=0pt,align=left,nosep}


  %% Some new commands
  \newcommand{\E}[1]{E[ #1 ]}
  \newcommand{\VAR}[1]{\mathrm{Var} [ #1 ]}
  \newcommand{\LAS}{\mathrm{LAS}}
  \newcommand{\mat}[1]{\bm{#1}}
  \newcommand{\proglang}[1]{\textsf{#1}}
  \newcommand{\pkg}[1]{\textbf{#1}}
  \newcommand{\code}[1]{\texttt{#1}}

  \bibliographystyle{plainnat}


\begin{document}

\maketitle

\section{Introduction}
\label{sec:introduction}

\proglang{R} already includes functions to compute the probability
density function (pdf), the cumulative distribution function (cdf) and
the quantile function of a fair number of probability laws, as well as
functions to generate variates from these laws. For some root
\code{foo}, the utility functions are named \code{dfoo}, \code{pfoo},
\code{qfoo} and \code{rfoo}, respectively.

The \pkg{actuar} package provides \code{d}, \code{p}, \code{q} and
\code{r} functions for a large number of continuous probability laws
useful for loss severity modeling, as well as zero-truncated and
zero-modified extensions to the most usual discrete probability laws
used in loss frequency modeling.


\section{Continuous distributions}
\label{sec:continuous}

The \pkg{actuar} package provides \code{d}, \code{p}, \code{q} and
\code{r} functions for all the probability laws found in Appendix A of
\cite{LossModels4e} and not already present in base \proglang{R},
excluding the inverse Gaussian and log-$t$ but including the loggamma
distribution \citep{HoggKlugman}. They mostly fall under the unmbrella
of extreme value distributions.

\autoref{tab:continuous} lists the supported distributions as
named in \cite{LossModels4e} along with the root names of the
\proglang{R} functions. For reference, \autoref{sec:app:continuous}
also gives for every distribution the pdf, the cdf and the name of the
argument corresponding to each parameter in the parametrization of
\cite{LossModels4e}. One will note that by default all functions
(except those for the Pareto distribution) use a rate parameter equal
to the inverse of the scale parameter. This differs from
\cite{LossModels4e} but is better in line with the functions for the
gamma, exponential and Weibull distributions in base \proglang{R}.

\begin{table}
  \centering
  \begin{tabular}{lll}
    \toprule
    Family & Distribution & Root \\
    \midrule
    Transformed beta  & Transformed beta & \code{trbeta} \\
                      & Burr & \code{burr} \\
                      & Loglogistic & \code{llogis} \\
                      & Paralogistic & \code{paralogis} \\
                      & Generalized Pareto & \code{genpareto} \\
                      & Pareto & \code{pareto} \\
                      & Inverse Burr & \code{invburr} \\
                      & Inverse Pareto & \code{invpareto} \\
                      & Inverse paralogistic & \code{invparalogis} \\
    \midrule
    Transformed gamma & Transformed gamma & \code{trgamma} \\
                      & Inverse transformed gamma & \code{invtrgamma} \\
                      & Inverse gamma & \code{invgamma} \\
                      & Inverse Weibull & \code{invweibull} \\
                      & Inverse exponential & \code{invexp} \\
    \midrule
    Other             & Loggamma & \code{lgamma} \\
                      & Single parameter Pareto & \code{pareto1} \\
                      & Generalized beta & \code{genbeta} \\
    \bottomrule
  \end{tabular}
  \caption{Probability laws supported by \pkg{actuar} classified by
    family and root names of the \proglang{R} functions.}
  \label{tab:continuous}
\end{table}

In addition to the \code{d}, \code{p}, \code{q} and \code{r}
functions, the package provides \code{m}, \code{lev} and \code{mgf}
functions to compute, respectively, theoretical raw moments
\begin{equation}
  \label{eq:def:moment}
  m_k = \E{X^k},
\end{equation}
theoretical limited moments
\begin{equation}
  \label{eq:def:limited-moment}
  \E{(X \wedge x)^k} = \E{\min(X, x)^k}
\end{equation}
and the moment generating function
\begin{equation}
  \label{eq:def:mgf}
  M_X(t) = \E{e^{tX}},
\end{equation}
when it exists. Every probability law of \autoref{tab:continuous}
is supported, plus the following ones: beta, exponential, chi-square,
gamma, lognormal, normal (no \code{lev}), uniform and Weibull of base
\proglang{R} and the inverse Gaussian distribution of package
\pkg{SuppDists} \citep{SuppDists}. The \code{m} and \code{lev}
functions are especially useful with estimation methods based on the
matching of raw or limited moments; see the \code{"lossdist"} vignette
for their empirical counterparts. The \code{mgf} functions come in
handy to compute the adjustment coefficient in ruin theory; see the
\code{"risk"} vignette.


\section{Phase-type distributions}
\label{sec:phase-type}

In addition to the 17 distributions of \autoref{tab:continuous},
the package provides support for a family of distributions deserving a
separate presentation. Phase-type distributions \citep{Neuts_81} are
defined as the distribution of the time until absorption of continuous
time, finite state Markov processes with $m$ transient states and one
absorbing state. Let
\begin{equation}
  \label{eq:Markov-transition-matrix}
  \mat{Q} =
  \begin{bmatrix}
    \mat{T} & \mat{t} \\
    \mat{0} & 0
  \end{bmatrix}
\end{equation}
be the transition rates matrix (or intensity matrix) of such a process
and let $(\pmb{\pi}, \pi_{m + 1})$ be the initial probability vector.
Here, $\mat{T}$ is an $m \times m$ non-singular matrix with $t_{ii} <
0$ for $i = 1, \dots, m$ and $t_{ij} \geq 0$ for $i \neq j$, $\mat{t}
= - \mat{T} \mat{e}$ and $\mat{e}$ is a column vector with all
components equal to 1. Then the cdf of the time until absorption
random variable with parameters $\pmb{\pi}$ and $\mat{T}$ is
\begin{equation}
  \label{eq:cdf-phtype}
  F(x) =
  \begin{cases}
    \pi_{m + 1}, & x = 0, \\
    1 - \pmb{\pi} e^{\mat{T} x} \mat{e}, & x > 0
  \end{cases}
\end{equation}
where
\begin{equation}
  \label{eq:matrix-exponential}
  e^{\mat{M}} = \sum_{n = 0}^\infty \frac{\mat{M}^n}{n!}
\end{equation}
is the matrix exponential of matrix $\mat{M}$.

The exponential, the Erlang (gamma with integer shape parameter) and
discrete mixtures thereof are common special cases of phase-type
distributions.

The package provides \code{d}, \code{p}, \code{r}, \code{m} and
\code{mgf} functions for phase-type distributions. The root is
\code{phtype} and parameters $\mat{\pi}$ and $\mat{T}$ are named
\code{prob} and \code{rates}, respectively. For the package, function
\code{pphtype} is central to the evaluation of the probability of
ruin; see \code{?ruin} and the \code{"risk"} vignette.


\section{Discrete distributions}
\label{sec:discrete}

The package introduces support functions for counting distributions
commonly used in loss frequency modeling. A counting distribution is a
discrete distribution defined on the non-negative integers
$0, 1, 2, \dots$.

Let $N$ be the counting random variable. We denote $p_k$ the
probability that the random variable $N$ takes the value $k$, that is:
\begin{equation*}
  p_k = \Pr[N = k].
\end{equation*}

\citet{LossModels4e} classify counting distributions in two main
classes. A discrete random variable is a member of the $(a, b, 0)$
class of distributions if there exists constants $a$ and $b$ such that
\begin{equation*}
  \frac{p_k}{p_{k - 1}} = a + \frac{b}{k}, \quad k = 1, 2, \dots.
\end{equation*}
The probability at zero, $p_0$ is set such that the probabilities sum
to $1$. The members of this class are the Poisson, the binomial, the
negative binomial and its special case, the geometric. These
distributions are all well supported in base \proglang{R} with
\code{d}, \code{p}, \code{q} and \code{r} functions.

The second class of distributions is the $(a, b, 1)$ class. A discrete
random variable is a member of the $(a, b, 0)$ class of distributions
if there exists constants $a$ and $b$ such that
\begin{equation*}
  \frac{p_k}{p_{k - 1}} = a + \frac{b}{k}, \quad k = 2, 3, \dots.
\end{equation*}
One will note that recursions start at $k = 2$ for the $(a, b, 1)$
class. Therefore, the probability at zero can be any arbitrary number
$0 \leq p_0 \leq 1$.

Setting $p_0 = 0$ defines a subclass of so-called
\emph{zero-truncated} distributions. The members of this subclass are
the zero-truncated Poisson, the zero-truncated binomial, the
zero-truncated negative binomial and the zero-truncated geometric.

Let $p(k)$ denote the probability mass function (pmf) of a member of
the $(a, b, 0)$ class of distributions, and let $p^T(k)$ denote the
pmf of a corresponding member of the $(a, b, 1)$ class. The cumulative
distributions functions are $P(k)$ and $P^T(k)$.

Zero-truncated distributions have probability mass function
$p^T(0) = 0$ and
\begin{equation*}
  p^T(k) = \frac{p(k)}{1 - p_0}, \quad k = 1, 2, \dots.
\end{equation*}
The distribution function is, for all $k = 0, 1, 2, \dots$,
\begin{equation*}
  P^T(k)
  = \frac{P(k) - P(0)}{1 - P(0)}
  = \frac{P(k) - p_0}{1 - p_0},
\end{equation*}
or, alternatively, the survival function $\bar{P}^T(k) = 1 - P^T(k)$
is
\begin{equation*}
  \bar{P}^T(k)
  = \frac{\bar{P}(k)}{\bar{P}(0)}
  = \frac{\bar{P}(k)}{1 - p_0}.
\end{equation*}

Package \pkg{actuar} introduces \code{d}, \code{p}, \code{q} and
\code{r} functions for the zero-truncated Poisson, zero-truncated
binomial, zero-truncated negative binomial and zero-truncated
geometric distributions. \autoref{tab:discrete} lists the root names
of the functions.

\begin{table}
  \centering
  \begin{tabular}{ll}
    \toprule
    Distribution & Root \\
    \midrule
    Zero-truncated Poisson & \code{ztpois} \\
    Zero-truncated binomial & \code{ztbinom} \\
    Zero-truncated negative binomial & \code{ztnbinom} \\
    Zero-truncated geometric & \code{ztgeom} \\
    Logarithmic & \code{logarithmic} \\
    \addlinespace[6pt]
    Zero-modified Poisson & \code{zmpois} \\
    Zero-modified binomial & \code{zmbinom} \\
    Zero-modified negative binomial & \code{zmnbinom} \\
    Zero-modified geometric & \code{zmgeom} \\
    Zero-modified logarithmic & \code{zmlogarithmic} \\
    \bottomrule
  \end{tabular}
  \caption{Members of the $(a, b, 1)$ class of discrete distributions
    supported by \pkg{actuar} and root names of the \proglang{R}
    functions.}
  \label{tab:discrete}
\end{table}

Also included in \autoref{tab:discrete} is the logarithmic (or
log-series) distribution. The logarithmic distribution with parameter
$\theta$ has pmf
\begin{equation*}
  p(k) = \frac{a \theta^x}{k}, \quad k = 1, 2, \dots,
\end{equation*}
with $a = -1/\log(1 - \theta)$ and for $0 \leq \theta < 1$. This is
the standard parametrization in the literature
\citep{Johnson:discrete:2005}.

The logarithmic distribution is always defined on the strictly
positive integers. As such, it is not qualifed as ``zero-truncated'',
but it nevertheless belongs to the $(a, b, 1)$ class of distributions,
more specifically to the subclass with $p_0 = 0$.

Actually, the logarithmic distribution is the limiting case of the
zero-truncated negative binomial distribution with size parameter
equal to zero. Note that in this context and using the parametrization
above, we have $\theta = 1 - p$, where $p$ is the probability of
success for the zero-truncated negative binomial. This differs from
the presentation in \citet{LossModels4e}.

Another subclass of the $(a, b, 1)$ class of distributions is obtained
by setting $p_0$ to some arbitrary number $p_0^M$ subject to
$0 < p_0^M \leq 1$. The members of this subclass are called
\emph{zero-modified} distributions. Zero-modified distributions are
discrete mixtures between a degenerate distribution at zero and the
corresponding distribution from the $(a, b, 0)$ class.

Let $p^M(k)$ and $P^M(k)$ denote the probability mass function and
cumulative distribution function, respectively, of a zero-modified
distribution. Seen as a mixture, a zero-modified distribution has pmf
\begin{equation}
  \label{eq:mixture}
  p^M(k) = \left(1 - \frac{1 - p_0^M}{1 - p0} \right) \mathbb{1}_{\{k = 0\}}(k)
  + \frac{1 - p_0^M}{1 - p0} p(k),
\end{equation}
where $\mathbb{1}_{\mathcal{A}}(k)$ is the pmf of a degenerate
distribution placing all its probability when $k \in \mathcal{A}$. The
pmf can also be expressed as $p^M(0) = p_0^M$ and
\begin{equation*}
  p^M(k) = (1 - p_0^M)\, \frac{p(k)}{1 - p_0}, \quad k = 1, 2, \dots.
\end{equation*}
The distribution function is, for all $k = 0, 1, 2, \dots$,
\begin{equation*}
  P^M(k)
  = p_0^M + (1 - p_0^M)\, \frac{P(k) - P(0)}{1 - P(0)}
  = p_0^M + (1 - p_0^M)\, \frac{P(k) - p_0}{1 - p_0}
\end{equation*}
and the survival function is
\begin{equation*}
  \bar{P}^T(k)
  = (1 - p_0^M)\, \frac{\bar{P}(k)}{\bar{P}(0)}
  = (1 - p_0^M)\, \frac{\bar{P}(k)}{1 - p_0}.
\end{equation*}

The members of the subclass are the zero-modified Poisson,
zero-modified binomial, zero-modified negative binomial and
zero-modified geometric, together with the zero-modified logarithmic
as a limiting case of the zero-modified negative binomial.
\autoref{tab:discrete} lists the root names of the support functions
provided in \pkg{actuar}.

Quite obviously, zero-truncated distributions are zero-modified
distributions with $p_0^M = 0$. However, using the dedicated functions
in \proglang{R} will be more efficient.


\section{Special integrals}
\label{sec:special-integrals}

Many of the cumulative distribution functions of
\autoref{sec:app:continuous} are expressed in terms of incomplete
gamma function or the incomplete beta function.

The incomplete gamma function is
\begin{displaymath}
  \Gamma(\alpha; x) = \frac{1}{\Gamma(\alpha)}
  \int_0^x t^{\alpha - 1} e^{-t}\, dt, \quad \alpha > 0, x > 0,
\end{displaymath}
with
\begin{displaymath}
  \Gamma(\alpha) = \int_0^\infty t^{\alpha - 1} e^{-t}\, dt,
\end{displaymath}
whereas the (regularized) incomplete beta function is
\begin{displaymath}
  \beta(a, b; x) = \frac{1}{\beta(a, b)}
  \int\limits_0^x t^{a - 1} (1 - t)^{b - 1}\, dt, \quad a > 0, b > 0, 0 < x < 1,
\end{displaymath}
with
\begin{align*}
  \beta(a, b)
  &= \int_0^1 t^{a - 1} (1 - t)^{b - 1}\, dt \\
  &= \frac{\Gamma(a) \Gamma(b)}{\Gamma(a + b)}.
\end{align*}

\citet{LossModels4e} also introduce three other integrals that play a
role in extending the range of admissible values for limited expected
value functions.

Let
\begin{equation}
  \label{eq:gammaint}
  G(\alpha; x) = \int_x^\infty t^{\alpha - 1} e^{-t}\, dt
\end{equation}
for $\alpha$ real and $x > 0$. When $\alpha > 0$, we clearly have
\begin{equation*}
  \label{eq:gammaint:apos}
  G(\alpha; x) = \Gamma(a) [1 - \Gamma(\alpha; x)].
\end{equation*}
The integral is also defined for $\alpha \le 0$. Integration by parts
yields the relationship
\begin{equation*}
  G(\alpha; x) = -\frac{x^\alpha e^{-x}}{\alpha}
  + \frac{1}{\alpha} G(\alpha + 1; x).
\end{equation*}
This process can be repeated until $\alpha + k$ is a positive number,
in which case the right hand side can be evaluated with
\eqref{eq:gammaint:apos}. If $\alpha = 0, -1, -2, \dots$, this
calculation requires the value of
\begin{equation*}
  \label{eq:expint}
  G(0; x) = \int_x^\infty \frac{e^{-t}}{t}\, dt = E_1(x),
\end{equation*}
which is known in the literature as the \emph{exponential integral}
\citep{Abramowitz:1972}.

Neither \citet{LossModels4e} nor \citet{Abramowitz:1972} provide a
name for the integral \eqref{eq:gammaint}. For the needs of the
package, we dubbed it the \emph{gamma integral}.

Let also
\begin{equation}
  \label{eq:betaint}
  B(a, b; x) = \Gamma(a + b) \int_0^x t^{a-1} (1-t)^{b-1} dt
\end{equation}
for $a > 0$, $b \neq -1, -2, \dots$ and $0 < x < 1$. Again, it is
clear that when $b > 0$,
\begin{equation*}
  B(a, b; x) = \Gamma(a) \Gamma(b) \beta(a, b; x).
\end{equation*}
Of more interest here is the case where $b < 0$,
$b \neq -1, -2, \dots$ and $a > 1 + \lfloor -b\rfloor$. Integration by
parts of \eqref{eq:betaint} yields
\begin{equation}
  \label{eq:betaint:bneg}
  \begin{split}
    B(a, b; x)
    &= \displaystyle
    -\Gamma(a + b) \left[ \frac{x^{a-1} (1-x)^b}{b}
      + \frac{(a-1) x^{a-2} (1-x)^{b+1}}{b (b+1)} \right. \\
    &  \displaystyle\left.
      + \cdots + \frac{(a-1) \cdots (a-r) x^{a-r-1}
        (1-x)^{b+r}}{b (b+1) \cdots (b+r)} \right] \\
    &   \displaystyle
    + \frac{(a-1) \cdots (a-r-1)}{b (b+1) \cdots (b+r)}
    \Gamma(a-r-1) \\
    &  \times \Gamma(b+r+1) \beta(a-r-1, b+r+1),
  \end{split}
\end{equation}
where $r = \lfloor -b\rfloor$. This expression is little found in the
literature outside of \citet{LossModels4e}. For the needs of
\pkg{actuar}, we dubbed \eqref{eq:betaint} the \emph{beta
  integral}.

The package contains functions to compute the gamma, exponential
and beta integrals. They are mostly used at the \proglang{C} level
to evaluate the limited expected value for distributions of the
transformed beta and transformed gamma families. The package also
provides the R interfaces \code{gammaint}, \code{expint} and
\code{betaint}, however these functions are not exported.


\section{Implementation}
\label{sec:implementation}

The core of all the functions presented in this documents is written
in \proglang{C} for speed.

The distribution function of the continuous distributions use
\code{pbeta} and \code{pgamma} to compute the incomplete beta and
incomplete gamma functions, respectively. The pmf, cdf and quantile
functions for the zero-truncated and zero-modified distributions use
the internal \proglang{R} functions for the corresponding standard
distributions for all but the trivial input values.

Generation of random variates from zero-truncated distributions uses
the simple inversion algorithm suggested by Peter Dalgaard on the
\code{r-help} mailing list on 1 May 2005\footnote{%
  \url{https://stat.ethz.ch/pipermail/r-help/2005-May/070680.html}}.
Let $u$ be a random number from a uniform distribution on $(p_0, 1)$.
Then
\begin{equation*}
  x = P^{-1}(u)
\end{equation*}
is distributed according to the zero-truncated version of the
distribution with cdf $P(k)$.

For zero-modified distributions, we generate variates from the
discrete mixture \eqref{eq:mixture} when $p_0^M \geq p_0$. When
$p_0^M < p_0$, we can use either of two methods:
\begin{enumerate*}[i)]
\item the classical rejection method with an envelope that differs
  from the target distribution only at zero (meaning that only zeros
  are rejected);
\item the inverse method on a restricted range explained above.
\end{enumerate*}
Which approach is faster depends on the relative speeds of the
standard random generation function and of the standard quantile
function, and on the proportion of zeros that are rejected using the
rejection algorithm. Based on the difference $p_0 - p_0^M$, we
determined cutoff points, different for each zero-modified
distribution, between the two methods.

The matrix exponential \proglang{C} routine needed in \code{dphtype}
and \code{pphtype} is based on \code{expm} from the package
\pkg{Matrix} \citep{Matrix}.

The C implementation of \code{expint} is based on code from the GNU
Software Library \citep{GSL}. The \proglang{C} code for
\code{gammaint} and \code{betaint} was written by the second author.


\appendix

\section{Main formulas for continuous distributions}
\label{sec:app:continuous}

This appendix gives the pdf and cdf of the probability laws appearing
in \autoref{tab:continuous} using the parametrization of
\cite{LossModels4e} and \cite{HoggKlugman}.

Unless otherwise stated all parameters are strictly positive and the
functions are defined for $x > 0$.

\subsection{Transformed beta family}
\label{sec:appendix:transformed-beta}

\subsubsection*{Transformed beta}

\begin{itemize}
\item Root: \code{trbeta}, \code{pearson6}
\item Parameters: \code{shape1} ($\alpha$),
      \code{shape2} ($\gamma$),
      \code{shape3} ($\tau$),
      \code{rate}   ($\lambda = 1/\theta$),
      \code{scale}  ($\theta$)
\end{itemize}

\begin{align*}
  f(x)
  &= \frac{\gamma u^\tau (1 - u)^\alpha}{x \beta
    (\alpha, \tau )},
    \qquad u = \frac{v}{1 + v},
    \qquad v = \left(\frac{x}{\theta} \right)^\gamma \\
  F(x)
  &= \beta (\tau, \alpha ; u) \\
  \E{X^k}
  &= \frac{%
    \theta^k \Gamma(\tau+k/\gamma) \Gamma(\alpha-k/\gamma)}{%
    \Gamma(\alpha) \Gamma(\tau)},
    \qquad -\tau\gamma < k < \alpha\gamma \\
  \E{(X \wedge x)^k}
  &= \frac{%
    \theta^k \Gamma(\tau+k/\gamma) \Gamma(\alpha-k/\gamma)}{%
    \Gamma(\alpha) \Gamma(\tau)}
    \beta(\tau+k/\gamma, \alpha-k/\gamma; u) \\
  &\phantom{=} + x^k (1 - F(x)),
    \qquad k > -\tau\gamma
\end{align*}

\subsubsection*{Burr}

\begin{itemize}
\item Root: \code{burr}
\item Parameters: \code{shape1} ($\alpha$),
      \code{shape2} ($\gamma$),
      \code{rate}   ($\lambda = 1/\theta$),
      \code{scale}  ($\theta$)
\end{itemize}

\begin{align*}
  f(x)
  &= \frac{\alpha \gamma u^\alpha (1 - u)}{x},
    \qquad u = \frac{1}{1 + v},
    \qquad v = \left( \frac{x}{\theta} \right)^\gamma \\
  F(x)
  &= 1 - u^\alpha \\
  \E{X^k}
  &= \frac{%
    \theta^k \Gamma(1+k/\gamma) \Gamma(\alpha-k/\gamma)}{%
    \Gamma(\alpha)},
    \qquad -\gamma < k < \alpha\gamma \\
  \E{(X \wedge x)^k}
  &= \frac{%
    \theta^k \Gamma(1+k/\gamma) \Gamma(\alpha-k/\gamma)}{%
    \Gamma(\alpha)}
    \beta(1+k/\gamma, \alpha-k/\gamma; 1-u) \\
  &\phantom{=} + x^k u^\alpha,
    \qquad k > -\gamma
\end{align*}

\subsubsection*{Loglogistic}

\begin{itemize}
\item Root: \code{llogis}
\item Parameters: \code{shape} ($\gamma$),
      \code{rate}   ($\lambda = 1/\theta$),
      \code{scale}  ($\theta$)
\end{itemize}

\begin{align*}
  f(x)
  &= \frac{\gamma u (1 - u)}{x},
    \qquad u = \frac{v}{1 + v},
    \qquad v = \left( \frac{x}{\theta} \right)^\gamma \\
  F(x)
  &= u \\
  \E{X^k}
  &= \theta^k \Gamma(1+k/\gamma) \Gamma(1-k/\gamma),
    \qquad -\gamma < k < \gamma \\
  \E{(X \wedge x)^k}
  &= \theta^k \Gamma(1+k/\gamma) \Gamma(1-k/\gamma)
    \beta(1+k/\gamma, 1-k/\gamma; u) \\
  &\phantom{=} + x^k (1 - u),
    \qquad k > -\gamma
\end{align*}

\subsubsection*{Paralogistic}

\begin{itemize}
\item Root: \code{paralogis}
\item Parameters: \code{shape} ($\alpha$),
      \code{rate}   ($\lambda = 1/\theta$),
      \code{scale}  ($\theta$)
\end{itemize}


\begin{align*}
  f(x)
  &= \frac{\alpha^2 u^\alpha (1 - u)}{x},
    \qquad u = \frac{1}{1 + v},
    \qquad v = \left( \frac{x}{\theta} \right)^\alpha \\
  F(x)
  &= 1 - u^\alpha \\
  \E{X^k}
  &= \frac{%
    \theta^k \Gamma(1+k/\alpha) \Gamma(\alpha-k/\alpha)}{%
    \Gamma(\alpha)},
    \qquad -\alpha < k < \alpha^2 \\
  \E{(X \wedge x)^k}
  &= \frac{%
    \theta^k \Gamma(1+k/\alpha) \Gamma(\alpha-k/\alpha)}{%
    \Gamma(\alpha)}
    \beta(1+k/\alpha, \alpha-k/\alpha; 1-u) \\
  &\phantom{=} + x^k u^\alpha,
    \qquad k > -\alpha

\end{align*}

\subsubsection*{Generalized Pareto}

\begin{itemize}
\item Root: \code{genpareto}
\item Parameters: \code{shape1} ($\alpha$),
      \code{shape2} ($\tau$),
      \code{rate}   ($\lambda = 1/\theta$),
      \code{scale}  ($\theta$)
\end{itemize}

\begin{align*}
  f(x)
  &= \frac{u^\tau (1 - u)^\alpha}{x \beta (\alpha, \tau )},
    \qquad u = \frac{v}{1 + v},
    \qquad v = \frac{x}{\theta} \\
  F(x)
  &= \beta (\tau, \alpha ; u) \\
  \E{X^k}
  &= \frac{%
    \theta^k \Gamma(\tau+k) \Gamma(\alpha-k)}{%
    \Gamma(\alpha) \Gamma(\tau)},
    \qquad -\tau < k < \alpha \\
  \E{(X \wedge x)^k}
  &= \frac{%
    \theta^k \Gamma(\tau+k) \Gamma(\alpha-k)}{%
    \Gamma(\alpha) \Gamma(\tau)}
    \beta(\tau+k, \alpha-k; u) \\
  &\phantom{=} + x^k (1 - F(x)),
    \qquad k > -\tau
\end{align*}

\subsubsection*{Pareto}

\begin{itemize}
\item Root: \code{pareto}, \code{pareto2}
\item Parameters: \code{shape} ($\alpha$),
      \code{scale}  ($\theta$)
\end{itemize}

\begin{align*}
  f(x)
  &= \frac{\alpha u^\alpha (1 - u)}{x},
    \qquad u = \frac{1}{1 + v},
    \qquad v = \frac{x}{\theta} \\
  F(x)
  &= 1 - u^\alpha \\
  \E{X^k}
  &= \frac{%
    \theta^k \Gamma(1+k) \Gamma(\alpha-k)}{%
    \Gamma(\alpha)},
    \qquad -1 < k < \alpha \\
  \E{(X \wedge x)^k}
  &= \frac{%
    \theta^k \Gamma(1+k) \Gamma(\alpha-k)}{%
    \Gamma(\alpha)}
    \beta(1+k, \alpha-k; 1-u) \\
  &\phantom{=} + x^k u^\alpha,
    \qquad k > -1
\end{align*}

\subsubsection*{Inverse Burr}

\begin{itemize}
\item Root: \code{invburr}
\item Parameters: \code{shape1} ($\tau$),
      \code{shape2} ($\gamma$),
      \code{rate}   ($\lambda = 1/\theta$),
      \code{scale}  ($\theta$)
\end{itemize}

\begin{align*}
  f(x)
  &= \frac{\tau \gamma u^\tau (1 - u)}{x},
    \qquad u = \frac{v}{1 + v},
    \qquad v = \left( \frac{x}{\theta} \right)^\gamma \\
  F(x)
  &= u^\tau \\
  \E{X^k}
  &= \frac{%
    \theta^k \Gamma(\tau+k/\gamma) \Gamma(1-k/\gamma)}{%
    \Gamma(\tau)},
    \qquad -\gamma < k < \alpha\gamma \\
  \E{(X \wedge x)^k}
  &= \frac{%
    \theta^k \Gamma(\tau+k/\gamma) \Gamma(1-k/\gamma)}{%
    \Gamma(\tau)}
    \beta(\tau+k/\gamma, 1-k/\gamma; u) \\
  &\phantom{=} + x^k (1-u^\tau),
    \qquad k > -\tau\gamma
\end{align*}

\subsubsection*{Inverse Pareto}

\begin{itemize}
\item Root: \code{invpareto}
\item Parameters: \code{shape} ($\tau$),
      \code{scale}  ($\theta$)
\end{itemize}

\begin{align*}
  f(x)
  &= \frac{\tau u^\tau (1 - u)}{x},
    \qquad u = \frac{v}{1 + v},
    \qquad v = \frac{x}{\theta} \\
  F(x)
  &= u^\tau \\
  \E{X^k}
  &= \frac{%
    \theta^k \Gamma(\tau+k) \Gamma(1-k)}{%
    \Gamma(\tau)},
    \qquad -\tau < k < 1 \\
  \E{(X \wedge x)^k}
  &= \theta^k \tau \int_0^u y^{\tau+k-1} (1 - y)^{-k}/, dy
  &\phantom{=} + x^k (1-u^\tau),
    \qquad k > -\tau
\end{align*}

\subsubsection*{Inverse paralogistic}

\begin{itemize}
\item Root: \code{invparalogis}
\item Parameters: \code{shape} ($\tau$),
      \code{rate}   ($\lambda = 1/\theta$),
      \code{scale}  ($\theta$)
\end{itemize}

\begin{align*}
  f(x)
  &= \frac{\tau^2 u^\tau (1 - u)}{x},
    \qquad u = \frac{v}{1 + v},
    \qquad v = \left(\frac{x}{\theta} \right)^\tau \\
  F(x)
  &= u^\tau \\
  \E{X^k}
  &= \frac{%
    \theta^k \Gamma(\tau+k/\tau) \Gamma(1-k/\tau)}{%
    \Gamma(\tau)},
    \qquad -\tau^2 < k < \tau \\
  \E{(X \wedge x)^k}
  &= \frac{%
    \theta^k \Gamma(\tau+k/\tau) \Gamma(1-k/\tau)}{%
    \Gamma(\tau)}
    \beta(\tau+k/\tau, 1-k/\tau; u) \\
  &\phantom{=} + x^k (1-u^\tau),
    \qquad k > -\tau^2
\end{align*}

\subsection{Transformed gamma family}
\label{sec:appendix:transformed-gamma}

\subsubsection*{Transformed gamma}

\begin{itemize}
\item Root: \code{trgamma}
\item Parameters: \code{shape1} ($\alpha$),
      \code{shape2} ($\tau$),
      \code{rate}   ($\lambda = 1/\theta$),
      \code{scale}  ($\theta$)
\end{itemize}

\begin{align*}
  f(x) &= \frac{\tau u^\alpha e^{-u}}{x \Gamma(\alpha)},
  \qquad u = \left( \frac{x}{\theta} \right)^\tau \\
  F(x) &= \Gamma (\alpha ; u)
\end{align*} \\

\subsubsection*{Inverse transformed gamma}

\begin{itemize}
\item Root: \code{invtrgamma}
\item Parameters: \code{shape1} ($\alpha$),
      \code{shape2} ($\tau$),
      \code{rate}   ($\lambda = 1/\theta$),
      \code{scale}  ($\theta$)
\end{itemize}

\begin{align*}
  f(x) &= \frac{\tau u^\alpha e^{-u}}{x\Gamma (\alpha)},
  \qquad u = \left( \frac{\theta}{x} \right)^\tau \\
  F(x) &= 1 - \Gamma (\alpha ; u)
\end{align*}

\subsubsection*{Inverse gamma}

\begin{itemize}
\item Root: \code{invgamma}
\item Parameters: \code{shape} ($\alpha$),
      \code{rate}   ($\lambda = 1/\theta$),
      \code{scale}  ($\theta$)
\end{itemize}

\begin{align*}
  f(x) &= \frac{u^\alpha e^{-u}}{x\Gamma (\alpha)},
  \qquad u = \frac{\theta}{x}\\
  F(x) &= 1 - \Gamma (\alpha ; u)
\end{align*}


\subsubsection*{Inverse Weibull}

\begin{itemize}
\item Root: \code{invweibull}, \code{lgompertz}
\item Parameters: \code{shape} ($\tau$),
      \code{rate}   ($\lambda = 1/\theta$),
      \code{scale}  ($\theta$)
\end{itemize}

\begin{align*}
  f(x) &= \frac{\tau u e^{-u}}{x},
  \qquad u = \left( \frac{\theta}{x} \right)^\tau \\
  F(x) &= e^{-u}
\end{align*}

\subsubsection*{Inverse exponential}

\begin{itemize}
\item Root: \code{invexp}
\item Parameters: \code{rate}   ($\lambda = 1/\theta$),
      \code{scale}  ($\theta$)
\end{itemize}

\begin{align*}
  f(x) &= \frac{u e^{-u}}{x},
  \qquad u = \frac{\theta}{x} \\
  F(x) &= e^{-u}
\end{align*}

\subsection{Other distributions}
\label{sec:appendix:other}

\subsubsection*{Loggamma}

\begin{itemize}
\item Root: \code{lgamma}
\item Parameters: \code{shapelog} ($\alpha$),
      \code{ratelog}   ($\lambda$)
\end{itemize}

\begin{align*}
  f(x) &= \frac{\lambda^\alpha (\ln x)^{\alpha - 1}}{%
    x^{\lambda + 1} \Gamma(\alpha)},
  \qquad x > 1 \\
  F(x) &= \Gamma( \alpha ; \lambda \ln x), \qquad x > 1
\end{align*}

\subsubsection*{Single parameter Pareto}

\begin{itemize}
\item Root: \code{pareto1}
\item Parameters: \code{shape} ($\alpha$),
      \code{min}   ($\theta$)
\end{itemize}

\begin{align*}
  f(x) &= \frac{\alpha
    \theta^\alpha}{x^{\alpha+1}}, \qquad x > \theta \\
  F(x) &= 1 - \left( \frac{\theta}{x} \right)^\alpha, \qquad x >
  \theta
\end{align*}

Although there appears to be two parameters, only $\alpha$ is a true
parameter. The value of $\theta$ is the minimum of the distribution
and is usually set in advance.

\subsubsection*{Generalized beta}

\begin{itemize}
\item Root: \code{genbeta}
\item Parameters: \code{shape1} ($\alpha$),
      \code{shape2} ($\beta$),
      \code{shape3} ($\tau$),
      \code{rate}   ($\lambda = 1/\theta$),
       \code{scale}  ($\theta$)
\end{itemize}

\begin{align*}
  f(x) &= \frac{\tau u^\alpha (1 - u)^{\beta - 1}}{x \beta (\alpha, \beta)},
  \qquad u = \left( \frac{x}{\theta} \right)^\tau,
  \qquad 0 < x < \theta \\
  F(x) &= \beta (\alpha, \beta ; u)
\end{align*}

\bibliography{actuar}


\end{document}

%%% Local Variables:
%%% mode: latex
%%% coding: utf-8
%%% TeX-master: t
%%% End:
